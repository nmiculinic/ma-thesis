\documentclass[times, utf8, diplomski, english]{fer}
\usepackage{booktabs}
\usepackage[hidelinks]{hyperref}
\usepackage{footnote}
\usepackage{graphicx}

\DeclareMathOperator*{\argmin}{\arg\!\min}
\DeclareMathOperator*{\argmax}{\arg\!\max}

\graphicspath{ {./figures/} }


\begin{document}

\thesisnumber{1572}
\title{ End-to-End Deep Learning Model for Base Calling of MinION Nanopore Reads}
\author{Neven Miculinić}
\maketitle
 
\izvornik

\zahvala{I would like to thank my mentor, Mile Šikić, for his patient guidance, encouragement
	and advice provided over the years.
	
I would also like to thank my family and friends for their
	continuous support.
	
In the end, honorable mentions go to Marko Ratković for his help with this thesis.	
}

\tableofcontents
\listoffigures
\listoftables

% !TEX root = ../main.tex

\chapter{Introduction}
\label{chap:Introduction}

% TODO: Modify as appropriately 
In recent years,  deep learning methods significantly improved the state-of-the-art in multiple domains such as computer vision, speech recognition, and natural language processing \cite{LeCun:1998:CNI:303568.303704}\cite{NIPS2012_4824}. 
In this thesis, we present application of deep learning in the field of  Bioinformatics for analysis of DNA sequencing data. 

DNA is a molecule that makes up the genetic material of a cell, and it is responsible for carrying the information needed for survival, growth, and reproduction of an organism. 
DNA is a long polymer of simple blocks called nucleotides connected together forming two spiraling strands to a structure called a double helix.  Possible nucleotide bases of a DNA strand are adenine, cytosine, guanine, thymine usually represented with letters A, C, G, and T. The order of these bases is what defines genetic code.

DNA sequencing is the process of determining this sequence of nucleotides. Originally sequencing was an expensive process, but during the last couple of decades, the price of sequencing has drastically decreased.  A significant breakthrough occurred in May 2015 with the release of MinION sequencer by Oxford Nanopore making DNA sequencing inexpensive and more available, even for small research teams. 

Base calling is a process assigning sequence of nucleotides (letters) to the raw data generated by the sequencing device. Simply put, it is a process of decoding the output from the sequencer.

\section{Objectives}

The objective of this thesis is try out novel approach in basecalling the raw sequence. We had good results with earlier R9 chemisty \citep{miculinic2017mincall} and we're experimenting with new approaches.

\section{Organization}
\chapter{Background}
\label{chap:background}

\section{Related work}
\chapter{Methods}
\chapter{Implementation}
\chapter{Results}
\chapter{Conclusion}

\bibliography{literatura}
\bibliographystyle{unsrtnat}
% \bibliographystyle{plainnat}

\begin{abstract}
In the MinION device, single-stranded DNA fragments move through nanopores, which causes drops in the electric current. The electric current is measured at each pore several thousand times per second. Each event is described by the mean and variance of the current and by event duration. This sequence of events is then translated into a DNA sequence by a base caller. Develop a base-caller for MinION nanopore sequencing platform using a deep learning architecture such as convolutional neural networks and recurrent neural networks. Instead of events, use current waveform at the input. Compare the accuracy with the state-of-the-art basecallers. For testing purposes use publicly, available datasets and Graphmap or Minimap 2 tools for aligning called reads on reference genomes.  Implement method using TensorFlow or similar library. The code should be documented and hosted on a publicly available Github repository.

\keywords{base calling, Oxford Nanopore Technologies, MinION, deep learning, seq2seq, convolutional neural network, residual network, CTC loss}
\end{abstract}

% TODO: Navedite naslov na hrvatskom jeziku.
\hrtitle{S kraja na kraj model dubokog učenja za određivanje očitanih baza dobivenih uređajem za sekvenciranje MinION}
\begin{sazetak}
    Unutar uređaja MinION, fragmenti jednostruke DNA prolaze kroz nanopore, što uzrokuje promjene u električnoj struji. Struja proizvedena na svakoj nanopori mjeri se nekoliko tisuća puta u sekundi. Svaki događaj opisan je srednjom vrijednosti i varijancom struje te svojim trajanjem. Postupak kojim se takav slijed događaja prevodi u niz nukleotida naziva se određivanje očitanih baza. Razviti alat za prozivanje baza za uređaj za sekvenciranje MinION koristeći modele dubokog učenje kao što su konvolucijske i povratne neuronske mreže. Umjesto događaja na ulazu koristi valni oblik struje. Usporediti dobivenu točnost s postojećim rješenjima. U svrhu testiranja koristiti javno dostupne skupove podataka i alate GraphMap ili Minimap 2 za poravnanje očitanja na referentni genom. Alat implementirati koristeći programsku biblioteku TensorFlow (ili neku sličnu). Programski kod treba biti dokumentiran i javno dostupan preko repozitorija GitHub.
\kljucnerijeci{određivanje baza, Oxford Nanopore Technologies, MinION, duboko učenje, prevođenje, konvolucijske neuronske mreže, rezidualne mreže, CTC gubitak}
\end{sazetak}

\end{document}